% ________________________________________________
% Layout
% ________________________________________________
\usepackage[notindex,nottoc]{tocbibind} % Inhaltsverzeichnisse erstellen
\usepackage{setspace} % für Zeilenabstand
\onehalfspacing % anderthalbzeiliger Abstand
\usepackage[
	left=2.5cm,
	right=2.5cm,
	top=2.5cm,
	bottom=2.5cm
]{geometry} %Ränder
\usepackage{amsthm}
\usepackage[
    skip=12pt, % Zeilenabstand nach Absatz
    indent=0pt % kein Einrücken zu Absatzbeginn
]{parskip}  
\setlength{\itemsep}{-6pt}

% ________________________________________________
% Schriften
% ________________________________________________ 
\usepackage{charter} 
\usepackage[scaled]{helvet} % Serifenlose Schrift wird in Helvetica geschrieben
\usepackage{courier} % mit \texttt für Code im Fließtext

% ________________________________________________
% Sprache
% ________________________________________________    
\usepackage[ngerman]{babel}     % Neue deutsche Rechtschreibung, Umlaute können geschrieben werden
\usepackage[utf8]{inputenc}     % direkte Angabe von Umlauten
\usepackage[T1]{fontenc}        % Silbentrennung bei Sonderzeichen

% ________________________________________________
% Format der Abbilung-/Tabellen Über-/Unterschriften
% ________________________________________________
\usepackage[
    labelsep=colon, % Doppelpunkt nach Tabelle/Abbildung
    justification=centering, % Mittig
    labelfont=bf, % Tabelle/Abbildung fett
]{caption} 

% Erstelle einen eingen Befehl für übernommene Tabellen/Abbildungen
\newcommand*{\captionsource}[2]{%
  \caption[{#1}]{%
    #1%
    \\
    \textbf{Quelle:} #2%
  }}

% ________________________________________________
% Mathe
% ________________________________________________  
\usepackage{amsmath} % Mathematik
\usepackage{marvosym} % enthält Symbole (wie €): https://ctan.mc1.root.project-creative.net/fonts/marvosym/doc/fonts/marvosym/marvodoc.pdf
\usepackage{amsfonts} % fonts for use in mathematics
\usepackage[np, autolanguage]{numprint} % Darstellung von Zahlen mit Einheiten

% Befehle für Referenzierung (Unterscheidung von Restriktion und Gleichung)
\newcommand{\myrestref}[1]{\hyperref[#1]{Restriktion (\ref*{#1})}}
\newcommand{\myeqref}[1]{\hyperref[#1]{Gleichung (\ref*{#1})}}

% ________________________________________________
% Farben
% ________________________________________________  
\usepackage[dvipsnames]{xcolor} % The option dvipsnames loads 68 colors, see https://ctan.org/pkg/xcolor?lang=de -> pdf, S.38
% Definition von eignenen Farben (diese Farben sind beispielsweise Farben der Universität)
\definecolor{luhblau}{RGB}{0,80,155}
\definecolor{luhmittelblau}{RGB}{153,185,216}
\definecolor{luhhellblau}{RGB}{204,220,235}
\definecolor{luhgruen}{RGB}{200,211,23}
\definecolor{luhdunkelgrau}{RGB}{153,153,153}
\definecolor{luhhellgrau}{RGB}{204,204,204}


% ________________________________________________
% Abbildungen
% ________________________________________________
\usepackage{graphics} % Einbindung von Abbildungen
\usepackage{graphicx} % Zusäztzliche Optionen beim Einbinden von Grafiken
\usepackage{pgfpages} % Einbindung von (mehrseitigen) PDFs
\usepackage{subcaption}
\usepackage{float}
\usepackage[percent]{overpic} % Bild über Bild

% ________________________________________________
% Abbildungen zeichnen mit Tikz
% ________________________________________________
\usepackage{pgfplots}

\pgfplotsset{compat=1.18} 
\usepackage{tikz}
\usetikzlibrary{positioning, arrows.meta, shapes}

% ________________________________________________
% Tabellen
% ________________________________________________
\usepackage{xltabular} % Tabellen mit Umbrüchen
\usepackage{booktabs} % Linien in Tabellen
\usepackage{multirow} % fasst mehrere Zeilen einer Tabelle zusammen 
\renewcommand{\arraystretch}{1.2} % Zeilenabstand in Tabellen

% ________________________________________________
% Listings (Darstellung von Code in der Arbeit) 
% ________________________________________________
\usepackage{listings}
\usepackage{listingsutf8}

% Einstellungen für die Darstellung des Codes
\lstset{frame=tb,
	basicstyle= \ttfamily \linespread{0.8} \small,
                keywordstyle=\color{black}\bfseries,
                stringstyle=\color{black}\ttfamily,
                commentstyle=\color{gray}\ttfamily,
                numbers=left,
                numberstyle=\color[rgb]{0,0.5,0.5}\fontfamily{pcr}\fontseries{m}\selectfont \scriptsize,
	            numberblanklines=false,
	            breaklines=true,
	            belowskip=\bigskipamount{},
	            escapeinside={<@}{@>}
}

% Python und C++ können als weit verbreitete Sprachen einfach geladen werden. 
% Die Keywords und Kommentare sind damit automatisch definiert und werden entsprechend erkannt.
% GAMS muss zunächst definiert werden, damit die Keyword und Kommentare richtig eingefärbt werden.
\lstdefinelanguage{GAMS}{
	morekeywords={
		ABORT , ACRONYM , ACRONYMS , ALIAS , ALL , AND , ASSIGN , BINARY , CARD , DISPLAY , EPS , EQ , EQUATION , EQUATIONS , GE , GT , INF , INTEGER , LE , LOOP , LT , MODEL , MODELS , NA , NE , NEGATIVE , NOT , OPTION , OPTIONS , OR , ORD , PARAMETER , PARAMETERS , POSITIVE , PROD , SCALAR , SCALARS , SET , SETS , SMAX , SMIN , SOS1 , SOS2 , SUM , SYSTEM , TABLE , VARIABLE , VARIABLES , XOR , YES , REPEAT , UNTIL , WHILE , IF , THEN , ELSE , SEMICONT , SEMIINT , FILE , FILES , PUT , PUTPAGE , PUTTL , PUTCLOSE , FREE , NO , SOLVE , FOR , ELSEIF , ABS , ARCTAN , CEIL , COS , ERROR , EXP , FLOOR , LOG , LOG10 , MAP , MAPVAL , MAX , MIN , MOD , NORMAL , POWER , ROUND , SIGN , SIN , SQR , SQRT , TRUNC , UNIFORM , LO , UP , FX , SCALE , PRIOR , PS , PW , TM , BM , CASE , DATE , IFILE , OFILE , PAGE , RDATE , RFILE , RTIME , SFILE , TITLE , /,PROD: },
	sensitive = false,
	morecomment=[f]*, %
	morecomment=[s]{\$ontext}{\$offtext},
	morestring=[b]”,
	morestring=[b]’
}

% Umlaute im Code (z.B. in Kommentaren) 
% Hinweis: Besser ist es, wenn der gesamte Code auf Englisch ist
\lstset{literate=%
	{Ö}{{\"O}}1
	{Ä}{{\"A}}1
	{Ü}{{\"U}}1
	{ß}{{\ss}}1
	{ü}{{\"u}}1
	{ä}{{\"a}}1
	{ö}{{\"o}}1
	{~}{{\textasciitilde}}1
}
\renewcommand{\lstlistlistingname}{Quellcodeverzeichnis}
\def\lstlistingautorefname{Quellcode} % definition des Begriffs für \autoref
\renewcommand{\lstlistingname}{Quellcode} % Ändern der Bezeichnung auf Deutsch

% ________________________________________________
% Pseudo Code in der Arbeit 
% ________________________________________________
\usepackage{algorithm} % float wrapper for algorithm (like table or figure)
\usepackage{algorithmic} % environment for writing pseudo code
\floatname{algorithm}{Algorithmus} % Ändern der Bezeichnung auf Deutsch
\def\algorithmautorefname{Algorithmus} % definition des Begriffs für \autoref
\renewcommand{\listalgorithmname}{Algorithmusverzeichnis} % Umbenennung des Verzeichnisses auf Deutsch
\captionsetup[algorithm]{labelfont=bf,labelsep=colon} % fügt einen Doppelpunkt ein (wie Tabellen und Abbildungen)
\algsetup{linenodelimiter=\ }

% ________________________________________________
% ToDO Notes und Kommentare
% ________________________________________________
\setlength {\marginparwidth }{2cm} 
\usepackage[
	textwidth=2cm, % Größe des Kommentarfeldes
	textsize=scriptsize	% Textgröße der Kommentare	
]{todonotes}
\usepackage{comment}

% ________________________________________________
% Literatur und Zitieren (Deutsche Zitierweise)
% ________________________________________________
\usepackage{csquotes}
\usepackage[
	style=authoryear,			% Zitationsstil
	%firstinits=true,			% Anfangsbuchstaben der Vornamen abkürzen veraltet?
	maxcitenames=2,				% et. al. ab 3 Autoren
	maxbibnames=99,				% alle Authoren im Literaturverzeichnis aufführen
	doi=true,					% doi anzeigen
	isbn=false,                 % isbn nicht anzeigen
	url=false,                  % url nciht anzeigen (außer bei online)
	backend=biber
]{biblatex}
\addbibresource{Literatur.bib}	% Hier die BIB Datei einbinden
\DefineBibliographyStrings{ngerman}{andothers={et\ al\adddot}} % aus u.a. et al. machen

\DeclareLabeldate{%
  \field{date}
  \field{year}
  \field{eventdate}
  \field{origdate}
  \literal{nodate}
}

% ________________________________________________
% Verweise im Text
% ________________________________________________  
\usepackage{hyperref}