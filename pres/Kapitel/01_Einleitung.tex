% Ein neues Kapitel beginnt mit \section{}. Das Inhaltsverzeichnis wird automatisch aus den sections generiert.
\section{Kapitel (z.B. Problembeschreibung)}

% ________________________________________________
% Folie mit Stichpunkten
% ________________________________________________
% Eine Folie beginnt mit \begin{frame} und endet mit \end{frame}
\begin{frame}{Stichpunkte}
    \begin{itemize}
        \item Die Problembeschreibung kann z.B. mit Stichpunkten beschrieben werden. Dabei sollten ganze Sätze vermieden werden!
        \begin{itemize}
            \item Unterpunkte sind auch möglich
        \end{itemize}
    \end{itemize}
\end{frame}

% ________________________________________________
% Folie mit Abbildung
% ________________________________________________
\begin{frame}{Abbildung}
    \begin{itemize}
        \item Noch besser ist die Verwendung von Abbildungen bei der Problembeschreibung!
    \end{itemize}
    \centering
    \includegraphics[width=0.5\textwidth]{example-image}
\end{frame}

% ________________________________________________
% Folie mit Stichpunkten und Bild rechts
% ________________________________________________
% mit begin{columns}... \end{columns} lässt sich eine Folie in mehrere Teile aufteilen 
\begin{frame}{Stichpunkte Bild Rechts}
    \begin{columns}
        \begin{column}{0.4\textwidth}
        \centering
            \begin{itemize}
                \item Stichpunkt
                \begin{itemize}
                    \item Substichpunkt
                \end{itemize}
            \end{itemize}
        \end{column}
        \begin{column}{0.6\textwidth}
        % Selbstständig erstellte Abbildungen brauchen keine Quelle. Eine Bildunterschrift muss nur angegeben werden, wenn das Thema nicht aus der Folie hervorgeht.
            \includegraphics[width=\textwidth]{example-image}
        \end{column}
    \end{columns}
\end{frame}


% ________________________________________________
% Folie mit zwei Abbildungen 
% ________________________________________________
% Inhalte können auch nacheinander eingeblendet werden: Alle Inhalte, die nach \pause kommen, werden erst auf der nächsten 'slide' eingeblendet. Ein Frame in Latex kann somit mehrere Slides (=meherere Folien in der Pdf) erzeugen. \pause kann mehrmals in einem Frame genutzt werden. Das geht bei Abbildungen, Texten, Aufzählungen, Tabellen etc.
\begin{frame}{Zwei Abbildungen, nacheinander eingeblendet}
    \begin{columns}
        \begin{column}{0.5\textwidth}
            \includegraphics[width=\textwidth]{example-image}
            \captionsource{Hier könnte eine Bildunterschrift stehen}{Hier die Quelle als Vollbeleg (siehe Leitfaden), mit Seitenangabe}
            % \captionsource ist ein eigner Befehl (siehe preamble), wenn eine Unterschrift und eine Quelle gesetzt werden soll.
        \end{column}
        \pause
        \begin{column}{0.5\textwidth}
            \includegraphics[width=\textwidth]{example-image}
            \source{Hier die Variante ohne Bildunterschrift und nur mit Quelle als Vollbeleg (siehe Leitfaden), mit Seitenangabe}
            % \source ist ein eigner Befehl (siehe preamble), wenn nur eine Quelle gesetzt werden soll.
        \end{column}
        
    \end{columns}
\end{frame}

% ________________________________________________
% Folie mit Textbox
% ________________________________________________
\begin{frame}{Textboxen}
    \begin{block}{Überschrift}
        Hier ist der Text für den Textblock, zB. eine Aufgabe oder Definition könnte hier stehen.
    \end{block}  
    
    \begin{exampleblock}{Beispiel}
       Bei einer Instanz von $c=5$ und $z=3$ würde Bauer Bernd 145,5 GE erhalten.
    \end{exampleblock}
    
    \begin{exampleblock}{}
       Für einen Textblock ohne Überschrift einfach die \{\} leer lassen.
    \end{exampleblock}
    
\end{frame}


\begin{frame}{Zitat}
    \begin{block}{}
        {\textit{''Hier [...] ein Beispielzitat.''}} \\
        \hspace*\fill{-- Name et al. (Jahr)}
       
    \end{block}
    
    \begin{block}{What is Operations Research?}
        {\textit{''Operations Research is the study of how to form mathematical models of complex engineering and management problems and how to analyze them to gain insight about possible solutions''}} \\
        \hspace*\fill{-- Rardin (2001), S.1}
       
    \end{block}
\end{frame}